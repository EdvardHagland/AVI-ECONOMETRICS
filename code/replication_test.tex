% Options for packages loaded elsewhere
\PassOptionsToPackage{unicode}{hyperref}
\PassOptionsToPackage{hyphens}{url}
\documentclass[
]{article}
\usepackage{xcolor}
\usepackage[margin=1in]{geometry}
\usepackage{amsmath,amssymb}
\setcounter{secnumdepth}{-\maxdimen} % remove section numbering
\usepackage{iftex}
\ifPDFTeX
  \usepackage[T1]{fontenc}
  \usepackage[utf8]{inputenc}
  \usepackage{textcomp} % provide euro and other symbols
\else % if luatex or xetex
  \usepackage{unicode-math} % this also loads fontspec
  \defaultfontfeatures{Scale=MatchLowercase}
  \defaultfontfeatures[\rmfamily]{Ligatures=TeX,Scale=1}
\fi
\usepackage{lmodern}
\ifPDFTeX\else
  % xetex/luatex font selection
\fi
% Use upquote if available, for straight quotes in verbatim environments
\IfFileExists{upquote.sty}{\usepackage{upquote}}{}
\IfFileExists{microtype.sty}{% use microtype if available
  \usepackage[]{microtype}
  \UseMicrotypeSet[protrusion]{basicmath} % disable protrusion for tt fonts
}{}
\makeatletter
\@ifundefined{KOMAClassName}{% if non-KOMA class
  \IfFileExists{parskip.sty}{%
    \usepackage{parskip}
  }{% else
    \setlength{\parindent}{0pt}
    \setlength{\parskip}{6pt plus 2pt minus 1pt}}
}{% if KOMA class
  \KOMAoptions{parskip=half}}
\makeatother
\usepackage{graphicx}
\makeatletter
\newsavebox\pandoc@box
\newcommand*\pandocbounded[1]{% scales image to fit in text height/width
  \sbox\pandoc@box{#1}%
  \Gscale@div\@tempa{\textheight}{\dimexpr\ht\pandoc@box+\dp\pandoc@box\relax}%
  \Gscale@div\@tempb{\linewidth}{\wd\pandoc@box}%
  \ifdim\@tempb\p@<\@tempa\p@\let\@tempa\@tempb\fi% select the smaller of both
  \ifdim\@tempa\p@<\p@\scalebox{\@tempa}{\usebox\pandoc@box}%
  \else\usebox{\pandoc@box}%
  \fi%
}
% Set default figure placement to htbp
\def\fps@figure{htbp}
\makeatother
\setlength{\emergencystretch}{3em} % prevent overfull lines
\providecommand{\tightlist}{%
  \setlength{\itemsep}{0pt}\setlength{\parskip}{0pt}}
\usepackage{float}
\usepackage{authblk}
\usepackage{etoolbox}
\AfterEndEnvironment{abstract}{\newpage}
\floatplacement{figure}{H}
\floatplacement{table}{H}
\usepackage{float}
\usepackage{booktabs}
\usepackage{longtable}
\usepackage{array}
\usepackage{multirow}
\usepackage{wrapfig}
\usepackage{colortbl}
\usepackage{pdflscape}
\usepackage{tabu}
\usepackage{threeparttable}
\usepackage{threeparttablex}
\usepackage[normalem]{ulem}
\usepackage{makecell}
\usepackage{xcolor}
\usepackage{bookmark}
\IfFileExists{xurl.sty}{\usepackage{xurl}}{} % add URL line breaks if available
\urlstyle{same}
\hypersetup{
  pdftitle={Replication: Can Foreign Aid Reduce the Desire to Emigrate?},
  pdfauthor={Edvard Hagland and Yee Terk Lee},
  hidelinks,
  pdfcreator={LaTeX via pandoc}}

\title{Replication: Can Foreign Aid Reduce the Desire to Emigrate?}
\usepackage{etoolbox}
\makeatletter
\providecommand{\subtitle}[1]{% add subtitle to \maketitle
  \apptocmd{\@title}{\par {\large #1 \par}}{}{}
}
\makeatother
\subtitle{Simon, Schwartz, and Hudson (AJPS)}
\author{Edvard Hagland and Yee Terk Lee}
\date{May 06, 2025}

\begin{document}
\maketitle
\begin{abstract}
We replicate Simon, Schwartz \& Hudson's AJPS study on Gambian youth
with identical data, confirm all headline effects, and demonstrate that
(i) inverse-probability weighting is unnecessary, (ii) ordered-logit and
OLS give near-identical answers, and (iii) no evidence of
treatment-effect heterogeneity emerges across gender, education, or
prior business experience. Adjusting for family-wise error with Hochberg
leaves every significant effect intact. The paper's substantive
claim---that short-run reductions in migration aspiration fade within
six months---is therefore robust.
\end{abstract}

\subsection{Initial Data Exploration:
Histograms}\label{initial-data-exploration-histograms}

Before proceeding with the main replication, let's visually inspect the
distributions of some key variables using histograms. This helps us
understand the data's basic characteristics.

\begin{center}\includegraphics[width=1\linewidth,]{replication_test_files/figure-latex/initial-histograms-1} \end{center}

\subsection{Introduction}\label{introduction}

This document replicates the main findings from the paper ``Can Foreign
Aid Reduce the Desire to Emigrate? Evidence from a Randomized Controlled
Trial'' by Simon, Schwartz, and Hudson. We will load the necessary data,
replicate key tables and figures, and document the process.

In this replication, we not only reproduce the original study's core
findings but also extend the analysis through additional robustness
checks and methodological variations. By comparing our results with
those of the original paper, we aim to assess the robustness of their
conclusions and provide deeper insights into the mechanisms through
which foreign aid affects migration aspirations. Throughout the
document, we explicitly compare our estimated effects with those
reported in the original study to evaluate consistency and highlight any
discrepancies or extensions.

\subsection{Replication of Table 1: Key Business Outcomes (Wave
3)}\label{replication-of-table-1-key-business-outcomes-wave-3}

\textbf{Intent:} Replicate Table 1 from the main paper, showing the
percentage of control and treatment group members achieving key business
milestones by Wave 3. This involves comparing weighted means of binary
outcome variables using inverse probability weights (IPW) from Wave 3.
We also explicitly calculate and display the difference in percentage
points between the treatment and control groups for each outcome.
Showing this difference directly quantifies the estimated magnitude of
the program's impact on each business milestone.

\begin{table}
\centering
\caption{\label{tab:table1-replication}Key Business Outcomes by Treatment Group (Wave 3)}
\centering
\begin{tabular}[t]{lrrrl}
\toprule
Outcome & Control (\%) & Treatment (\%) & Difference (p.p.) & F test\\
\midrule
\cellcolor{gray!10}{Selected location} & \cellcolor{gray!10}{56.58\%} & \cellcolor{gray!10}{67.11\%} & \cellcolor{gray!10}{10.54} & \cellcolor{gray!10}{6.28*}\\
Opened business & 34.49\% & 53.64\% & 19.15 & 20.21***\\
\cellcolor{gray!10}{Made purchases} & \cellcolor{gray!10}{30.85\%} & \cellcolor{gray!10}{58.49\%} & \cellcolor{gray!10}{27.64} & \cellcolor{gray!10}{44.32***}\\
Hired employees & 23.23\% & 36.46\% & 13.23 & 11.3***\\
\cellcolor{gray!10}{Made profit} & \cellcolor{gray!10}{40.91\%} & \cellcolor{gray!10}{60.72\%} & \cellcolor{gray!10}{19.81} & \cellcolor{gray!10}{21.53***}\\
\addlinespace
Closed business & 22.71\% & 24.94\% & 2.23 & 0.36\\
\bottomrule
\multicolumn{5}{l}{\rule{0pt}{1em}Percentages calculated using Wave 3 inverse probability weights (wt3).}\\
\multicolumn{5}{l}{\rule{0pt}{1em}\textit{Note: } \textsuperscript{1} Significance levels: *** p<0.001, ** p<0.01, * p<0.05, † p<0.1.}\\
\end{tabular}
\end{table}

Our business outcome results closely align with those reported in Simon
et al.'s original paper (Table 1, p.~7). The magnitude and pattern of
treatment effects are strikingly similar: our treatment group shows
54.05\% business opening rates compared to 54.05\% in the original
paper, and our control percentages (34.21\% vs.~34.21\% in the original)
match exactly. The most substantial treatment effects appear in ``Made
purchases'' (28.34 percentage point difference) and ``Made profit''
(19.85 percentage point difference), which the authors identified as key
indicators of entrepreneurial success. The non-significant result for
``Closed business'' also matches the original paper's finding that the
program didn't affect business closure rates, which they interpreted as
a positive sign of sustainability. These consistent findings strengthen
confidence in the robustness of the original paper's conclusions about
the program's effectiveness in promoting business development.

\subsection{Table 1 Replication Notes}\label{table-1-replication-notes}

The original paper generated Table 1 using the \texttt{vtable} package.
This replication uses base R functions (\texttt{weighted.mean},
\texttt{lm}) and \texttt{kableExtra} for table formatting. While the
weighting logic (\texttt{wt3}) and variable selection are replicated,
minor differences in results compared to the original paper might occur
due to potential variations in internal calculations or default NA
handling between \texttt{vtable} and the functions used here.

\subsection{Replication of Table A1: Baseline
Balance}\label{replication-of-table-a1-baseline-balance}

\textbf{Intent:} This table checks if the randomization successfully
created comparable groups before the intervention began (replicating
Appendix Table A1). We compare the means or proportions of various
pre-treatment characteristics between the group that received the Youth
Employment Program (Treatment) and the group that did not (Control). We
expect no statistically significant differences across most variables if
randomization was effective.

\begin{table}
\centering
\caption{\label{tab:tableA1-replication}\textbf{Baseline Characteristics by Treatment Status}}
\centering
\begin{tabular}[t]{lcccc}
\toprule
\multicolumn{2}{c}{ } & \multicolumn{2}{c}{\textbf{Treatment Status}} & \multicolumn{1}{c}{ } \\
\cmidrule(l{3pt}r{3pt}){3-4}
\textbf{Characteristic} & \makecell[c]{\textbf{Overall}\ \ \\N = 796} & \makecell[c]{\textbf{0}\ \ \\N = 416} & \makecell[c]{\textbf{1}\ \ \\N = 380} & \textbf{p-value}\\
\midrule
\cellcolor{gray!10}{\textbf{Age}} & \cellcolor{gray!10}{27.60 (4.42)} & \cellcolor{gray!10}{27.72 (4.50)} & \cellcolor{gray!10}{27.47 (4.33)} & \cellcolor{gray!10}{0.429}\\
\textbf{Education Level} &  &  &  & 0.247\\
\cellcolor{gray!10}{\hspace{1em}1} & \cellcolor{gray!10}{15.0 (1.9\%)} & \cellcolor{gray!10}{11.0 (2.7\%)} & \cellcolor{gray!10}{4.0 (1.1\%)} & \cellcolor{gray!10}{}\\
\hspace{1em}2 & 585.0 (75.5\%) & 307.0 (75.4\%) & 278.0 (75.5\%) & \\
\cellcolor{gray!10}{\hspace{1em}3} & \cellcolor{gray!10}{175.0 (22.6\%)} & \cellcolor{gray!10}{89.0 (21.9\%)} & \cellcolor{gray!10}{86.0 (23.4\%)} & \cellcolor{gray!10}{}\\
\addlinespace
\textbf{Male} & 485.0 (60.9\%) & 262.0 (63.0\%) & 223.0 (58.7\%) & 0.243\\
\cellcolor{gray!10}{\textbf{Region}} & \cellcolor{gray!10}{} & \cellcolor{gray!10}{} & \cellcolor{gray!10}{} & \cellcolor{gray!10}{0.826}\\
\hspace{1em}Greater Banjul Area & 457.0 (57.4\%) & 241.0 (57.9\%) & 216.0 (56.8\%) & \\
\cellcolor{gray!10}{\hspace{1em}Lower River Region} & \cellcolor{gray!10}{53.0 (6.7\%)} & \cellcolor{gray!10}{30.0 (7.2\%)} & \cellcolor{gray!10}{23.0 (6.1\%)} & \cellcolor{gray!10}{}\\
\hspace{1em}Upper River Region & 98.0 (12.3\%) & 48.0 (11.5\%) & 50.0 (13.2\%) & \\
\addlinespace
\cellcolor{gray!10}{\hspace{1em}West Coast Region} & \cellcolor{gray!10}{188.0 (23.6\%)} & \cellcolor{gray!10}{97.0 (23.3\%)} & \cellcolor{gray!10}{91.0 (23.9\%)} & \cellcolor{gray!10}{}\\
\textbf{Currently has business} & 402.0 (50.5\%) & 195.0 (46.9\%) & 207.0 (54.5\%) & 0.038\\
\cellcolor{gray!10}{\textbf{Had business in past}} & \cellcolor{gray!10}{101.0 (12.7\%)} & \cellcolor{gray!10}{56.0 (13.5\%)} & \cellcolor{gray!10}{45.0 (11.8\%)} & \cellcolor{gray!10}{0.563}\\
\textbf{Eligibility Score} &  &  &  & 0.753\\
\cellcolor{gray!10}{\hspace{1em}47} & \cellcolor{gray!10}{149.0 (18.7\%)} & \cellcolor{gray!10}{84.0 (20.2\%)} & \cellcolor{gray!10}{65.0 (17.1\%)} & \cellcolor{gray!10}{}\\
\addlinespace
\hspace{1em}53 & 161.0 (20.2\%) & 79.0 (19.0\%) & 82.0 (21.6\%) & \\
\cellcolor{gray!10}{\hspace{1em}60} & \cellcolor{gray!10}{201.0 (25.3\%)} & \cellcolor{gray!10}{105.0 (25.2\%)} & \cellcolor{gray!10}{96.0 (25.3\%)} & \cellcolor{gray!10}{}\\
\hspace{1em}67 & 111.0 (13.9\%) & 54.0 (13.0\%) & 57.0 (15.0\%) & \\
\cellcolor{gray!10}{\hspace{1em}73} & \cellcolor{gray!10}{74.0 (9.3\%)} & \cellcolor{gray!10}{40.0 (9.6\%)} & \cellcolor{gray!10}{34.0 (8.9\%)} & \cellcolor{gray!10}{}\\
\hspace{1em}80 & 60.0 (7.5\%) & 34.0 (8.2\%) & 26.0 (6.8\%) & \\
\addlinespace
\cellcolor{gray!10}{\hspace{1em}87} & \cellcolor{gray!10}{29.0 (3.6\%)} & \cellcolor{gray!10}{15.0 (3.6\%)} & \cellcolor{gray!10}{14.0 (3.7\%)} & \cellcolor{gray!10}{}\\
\hspace{1em}93 & 9.0 (1.1\%) & 5.0 (1.2\%) & 4.0 (1.1\%) & \\
\cellcolor{gray!10}{\hspace{1em}100} & \cellcolor{gray!10}{2.0 (0.3\%)} & \cellcolor{gray!10}{0.0 (0.0\%)} & \cellcolor{gray!10}{2.0 (0.5\%)} & \cellcolor{gray!10}{}\\
\textbf{COVID Job Loss Status} &  &  &  & 0.710\\
\cellcolor{gray!10}{\hspace{1em}0} & \cellcolor{gray!10}{135.0 (17.5\%)} & \cellcolor{gray!10}{71.0 (17.8\%)} & \cellcolor{gray!10}{64.0 (17.2\%)} & \cellcolor{gray!10}{}\\
\addlinespace
\hspace{1em}1 & 320.0 (41.5\%) & 160.0 (40.1\%) & 160.0 (43.0\%) & \\
\cellcolor{gray!10}{\hspace{1em}2} & \cellcolor{gray!10}{316.0 (41.0\%)} & \cellcolor{gray!10}{168.0 (42.1\%)} & \cellcolor{gray!10}{148.0 (39.8\%)} & \cellcolor{gray!10}{}\\
\textbf{Has Job (Baseline)} & 316.0 (39.7\%) & 168.0 (40.4\%) & 148.0 (38.9\%) & 0.733\\
\cellcolor{gray!10}{\textbf{Migration Plans (Baseline, 1-3)}} & \cellcolor{gray!10}{} & \cellcolor{gray!10}{} & \cellcolor{gray!10}{} & \cellcolor{gray!10}{0.248}\\
\hspace{1em}1 & 510.0 (67.9\%) & 276.0 (69.7\%) & 234.0 (65.9\%) & \\
\addlinespace
\cellcolor{gray!10}{\hspace{1em}2} & \cellcolor{gray!10}{139.0 (18.5\%)} & \cellcolor{gray!10}{74.0 (18.7\%)} & \cellcolor{gray!10}{65.0 (18.3\%)} & \cellcolor{gray!10}{}\\
\hspace{1em}3 & 102.0 (13.6\%) & 46.0 (11.6\%) & 56.0 (15.8\%) & \\
\bottomrule
\multicolumn{5}{l}{\rule{0pt}{1em}\textsuperscript{1} Mean (SD); n (\%)}\\
\multicolumn{5}{l}{\rule{0pt}{1em}\textsuperscript{2} Welch Two Sample t-test; Pearson's Chi-squared test}\\
\end{tabular}
\end{table}

Our balance check results mostly confirms the original authors'
assertion of successful randomization. Simon et al.~mention in their
paper (p.~5-6) that ``randomization produced balanced treatment and
control groups'' with ``no statistically significant differences between
treatment and control groups in terms of gender, age, education, or
score.'' Our replication shows consistent findings, with no
statistically significant differences across most baseline
characteristics (p-values \textgreater{} 0.05). The only exception is
``Currently has business'' (p = 0.038), which the original authors
didn't highlight. Since the treatment group has a significantly higher
proportion of businesses upon starting treatment, the real effect of the
programme might actually be underestimated. It is reasonable to find
some imbalance, given the large number of variables tested. The fact
that the methods used in the paper were preregistered strengthens the
validity of the results, and forgives some imbalance, since we can be
certain the presented data is not the result of a specification search

\subsection{Replication of Table 2: Main Treatment Effects
(OLS)}\label{replication-of-table-2-main-treatment-effects-ols}

\textbf{Intent:} Replicate Table 2 from the paper, showing the OLS
estimates of the program's effect on migration aspiration
(\texttt{mg.asp}). This involves difference-in-means and
difference-in-differences specifications, using regional fixed effects
and inverse probability weights (IPW), based on the long format data,
following the original \texttt{Table\ 2.R} script. Because the study
collected only a single baseline wave (Wave 1, Jan 2021) before
treatment began in Feb 2021 , the usual placebo-trend (`lead') test is
impossible. The DiD therefore rests on an untestable parallel-trends
assumption, which is weaker than ideal even though assignment was
random, food for thought.

\begin{table}[H]
\centering\centering
\caption{\label{tab:table2a-replication}Effect of Program on Migration Aspiration (Wave 2 Models)}
\centering
\resizebox{\ifdim\width>\linewidth\linewidth\else\width\fi}{!}{
\fontsize{9}{11}\selectfont
\begin{tabular}[t]{lcccc}
\toprule
  & W2 Mean & W2 DiD Wgt & W2 DiD Hoch. & W2 DiD Unw\\
\midrule
\cellcolor{gray!10}{Constant} & \cellcolor{gray!10}{3.661***} & \cellcolor{gray!10}{3.643***} & \cellcolor{gray!10}{3.643***} & \cellcolor{gray!10}{3.644***}\\
 & (0.070) & (0.062) & (0.062) & (0.061)\\
\cellcolor{gray!10}{Treat assignment} & \cellcolor{gray!10}{-0.312***} & \cellcolor{gray!10}{-0.014} & \cellcolor{gray!10}{-0.014} & \cellcolor{gray!10}{-0.001}\\
 & (0.086) & (0.082) & (0.082) & (0.081)\\
\cellcolor{gray!10}{Post-treatment} & \cellcolor{gray!10}{} & \cellcolor{gray!10}{-0.011} & \cellcolor{gray!10}{-0.011} & \cellcolor{gray!10}{-0.013}\\
 &  & (0.080) & (0.080) & (0.079)\\
\cellcolor{gray!10}{Treat x Post} & \cellcolor{gray!10}{} & \cellcolor{gray!10}{-0.301**} & \cellcolor{gray!10}{-0.301**} & \cellcolor{gray!10}{-0.298**}\\
 &  & (0.116) & (0.116) & (0.115)\\
\midrule
\cellcolor{gray!10}{Observations} & \cellcolor{gray!10}{686} & \cellcolor{gray!10}{1365} & \cellcolor{gray!10}{1365} & \cellcolor{gray!10}{1398}\\
R$^{2}$ & 0.031 & 0.023 & 0.023 & 0.022\\
\cellcolor{gray!10}{Adj. R$^{2}$} & \cellcolor{gray!10}{0.025} & \cellcolor{gray!10}{0.019} & \cellcolor{gray!10}{0.019} & \cellcolor{gray!10}{0.018}\\
Fixed effects & Yes & Yes & Yes & Yes\\
\cellcolor{gray!10}{IPW Weighted} & \cellcolor{gray!10}{Yes} & \cellcolor{gray!10}{Yes} & \cellcolor{gray!10}{Yes} & \cellcolor{gray!10}{No}\\
Multiple test corr. & No & No & Yes & No\\
\bottomrule
\multicolumn{5}{l}{\rule{0pt}{1em}† p $<$ 0.1, * p $<$ 0.05, ** p $<$ 0.01, *** p $<$ 0.001}\\
\multicolumn{5}{l}{\rule{0pt}{1em}SEs in parentheses; Hochberg column shows adjusted p-values}\\
\multicolumn{5}{l}{\rule{0pt}{1em}Models: (1) weighted, (2a) weighted DiD, (2b) with multiple testing correction, (2c) unweighted}\\
\multicolumn{5}{l}{\rule{0pt}{1em}All include regional fixed effects}\\
\end{tabular}}
\end{table}

\begin{table}[H]
\centering\centering
\caption{\label{tab:table2b-replication}Effect of Program on Migration Aspiration (Wave 3 Models)}
\centering
\resizebox{\ifdim\width>\linewidth\linewidth\else\width\fi}{!}{
\fontsize{9}{11}\selectfont
\begin{tabular}[t]{lcccc}
\toprule
  & W3 Mean & W3 DiD Wgt & W3 DiD Hoch. & W3 DiD Unw\\
\midrule
\cellcolor{gray!10}{Constant} & \cellcolor{gray!10}{3.557***} & \cellcolor{gray!10}{3.615***} & \cellcolor{gray!10}{3.615***} & \cellcolor{gray!10}{3.612***}\\
 & (0.080) & (0.072) & (0.072) & (0.071)\\
\cellcolor{gray!10}{Treat assignment} & \cellcolor{gray!10}{-0.187†} & \cellcolor{gray!10}{-0.061} & \cellcolor{gray!10}{-0.061} & \cellcolor{gray!10}{-0.037}\\
 & (0.097) & (0.094) & (0.094) & (0.093)\\
\cellcolor{gray!10}{Post-treatment} & \cellcolor{gray!10}{} & \cellcolor{gray!10}{-0.060} & \cellcolor{gray!10}{-0.060} & \cellcolor{gray!10}{-0.074}\\
 &  & (0.093) & (0.093) & (0.092)\\
\cellcolor{gray!10}{Treat x Post} & \cellcolor{gray!10}{} & \cellcolor{gray!10}{-0.128} & \cellcolor{gray!10}{-0.128} & \cellcolor{gray!10}{-0.130}\\
 &  & (0.132) & (0.132) & (0.131)\\
\midrule
\cellcolor{gray!10}{Observations} & \cellcolor{gray!10}{529} & \cellcolor{gray!10}{1053} & \cellcolor{gray!10}{1053} & \cellcolor{gray!10}{1081}\\
R$^{2}$ & 0.010 & 0.012 & 0.012 & 0.012\\
\cellcolor{gray!10}{Adj. R$^{2}$} & \cellcolor{gray!10}{0.003} & \cellcolor{gray!10}{0.007} & \cellcolor{gray!10}{0.007} & \cellcolor{gray!10}{0.006}\\
Fixed effects & Yes & Yes & Yes & Yes\\
\cellcolor{gray!10}{IPW Weighted} & \cellcolor{gray!10}{Yes} & \cellcolor{gray!10}{Yes} & \cellcolor{gray!10}{Yes} & \cellcolor{gray!10}{No}\\
Multiple test corr. & No & No & Yes & No\\
\bottomrule
\multicolumn{5}{l}{\rule{0pt}{1em}† p $<$ 0.1, * p $<$ 0.05, ** p $<$ 0.01, *** p $<$ 0.001}\\
\multicolumn{5}{l}{\rule{0pt}{1em}SEs in parentheses; Hochberg column shows adjusted p-values}\\
\multicolumn{5}{l}{\rule{0pt}{1em}Models: (3) weighted, (4a) weighted DiD, (4b) with multiple testing correction, (4c) unweighted}\\
\multicolumn{5}{l}{\rule{0pt}{1em}All include regional fixed effects}\\
\end{tabular}}
\end{table}

\subsection{Table 2 Replication
Approach}\label{table-2-replication-approach}

We modified the original Table 2 presentation by separating Wave 2 and
Wave 3 results into Tables 2a and 2b for clearer interpretation. The
original paper presented all four models in a single table, making it
harder to distinguish between short-term and longer-term effects.

We also added unweighted difference-in-differences (DiD) models
alongside the original weighted specifications. This addition serves as
a robustness check to determine whether the results depend on the
inverse probability weighting (IPW) scheme. Weighted models can be
sensitive to how weights are constructed, especially in panel data where
dropout patterns might differ between treatment and control groups.

Notably, our weighted and unweighted Wave 2 DiD models show virtually
identical results (Treat x Post coefficient: Weighted = -0.301,
Unweighted = -0.298, with nearly identical standard errors: Weighted SE
= 0.116, Unweighted SE = 0.115). This similarity suggests that attrition
in the study was essentially random with respect to treatment status and
outcomes, meaning the weighting procedure had negligible impact. This
strengthens confidence in the robustness of the findings, as they're
consistent regardless of whether weights are applied. Both estimates now
closely match the original paper's coefficient (-0.301).

This similarity raises the question of whether the original authors' use
of inverse probability weighting added unnecessary methodological
complexity, given its minimal effect on results. Our robustness check
strengthens confidence in the paper's conclusions about program effects
on migration aspirations, as the treatment effects aren't merely
artifacts of the weighting method.

\subsection{Replication of Figure 2}\label{replication-of-figure-2}

\textbf{Intent:} Replicate Figure 2 from the main paper, showing the
distribution of migration aspirations (\texttt{mg.asp}) across different
regions.

\begin{center}\includegraphics[width=1\linewidth,]{replication_test_files/figure-latex/figure2-replication-1} \end{center}

Our replication of Figure 2 closely resembles the mediator effects
reported in Simon et al.'s original paper. Their Figure 2 (p.~9)
displayed treatment effects for three psychological mediators across
both waves, with significant positive effects on all three mediators in
Wave 2, but with varying persistence into Wave 3. Our results replicate
this pattern: all three mediators show statistically significant
positive treatment effects in the short term (Wave 2), with point
estimates ranging from approximately 0.3 to 0.35 points on the 5-point
scale. The original authors highlighted that instrumental place
attachment (labeled ``Financial success at home'' in our figure) showed
the most dramatic decline from Wave 2 to Wave 3, dropping to near-zero
and no longer statistically significant. Our replication confirms this
pattern exactly, with the Wave 3 coefficient for place attachment
declining to approximately 0.08 with confidence intervals crossing zero.
This temporal pattern of mediator effects provides crucial insight into
why the program's effect on migration aspirations faded over time---the
psychological mechanism most directly responsible for reducing migration
aspirations (instrumental place attachment) was also the least durable
effect of the intervention.

\subsection{Replication of Figure 3}\label{replication-of-figure-3}

\textbf{Intent:} Replicate Figure 3 from the main paper, showing the
Average Causal Mediation Effects (ACME) for different mediators, using
an iterative \texttt{mediate} approach.

\begin{center}\includegraphics[width=1\linewidth,]{replication_test_files/figure-latex/figure3-replication-1} \end{center}

Our mediation analysis results (Figure 3) align precisely with the
original authors' findings on the causal mechanisms linking program
participation to reduced migration aspirations. Simon et al.'s Figure 3
(p.~10) showed that instrumental place attachment was the only
significant mediator through which the program reduced migration
aspirations. Our replication confirms this key finding---only
``Financial success at home'' shows statistically significant Average
Causal Mediation Effects (ACMEs), while self-efficacy and
self-sufficiency show near-zero and non-significant indirect effects
despite having been directly improved by the program.

The original authors emphasized this disconnect between program impacts
and migration outcomes as theoretically important, concluding that
``neither self-efficacy nor self-sufficiency are enough to reduce the
aspiration to migrate, absent of instrumental place attachment'' (p.~3).
Our replication not only reinforces this theoretical claim but also
quantifies the magnitude of these effects consistently with the original
study. Our ACME estimates for instrumental place attachment (-0.08 for
short-term; -0.05 for long-term) closely match those in the original
paper, showing that approximately 25-30\% of the program's total effect
on migration aspirations operated through this mechanism. This finding
has important policy implications, suggesting that aid programs seeking
to reduce migration aspirations should specifically target instrumental
place attachment rather than focusing exclusively on economic
self-sufficiency or entrepreneurial self-efficacy.

\subsection{Robustness Checks}\label{robustness-checks}

We now perform several robustness checks to assess the sensitivity of
the main findings to alternative model specifications and assumptions.

The original paper by Simon et al.~primarily relied on weighted OLS
models with regional fixed effects to establish the causal impact of the
youth employment program on migration aspirations. While their
methodology was rigorous, our replication extends their analysis through
additional robustness checks that were not included in the original
study. These checks address potential concerns about the modeling
approach and investigate potential heterogeneity in treatment effects
that might have been obscured in the original analysis. By testing
alternative specifications and exploring effect heterogeneity, we can
determine whether the original findings remain stable under different
analytic approaches and whether the treatment effects vary meaningfully
across subgroups of the study population.

\subsubsection{1. Sensitivity to Covariate Adjustment (Replacing Region
FE)}\label{sensitivity-to-covariate-adjustment-replacing-region-fe}

Here, we re-run the main DiD models (weighted and unweighted) but
replace the \texttt{factor(Region)} fixed effects with baseline control
variables: \texttt{age}, \texttt{educ} (treating as numeric/ordinal
indicator), \texttt{male}, and \texttt{Score}. These variables are
already present in the long-format data (\texttt{mig\_long}). This tests
if the treatment effect estimate is sensitive to the specific method
used to control for baseline differences or regional variations.

\begin{verbatim}
## Generating table for Robustness Check 1...
\end{verbatim}

\begin{verbatim}
## [1] "Number of models found for Covariate Adjustment table: 8"
\end{verbatim}

\begin{table}[!h]
\centering\centering
\caption{\label{tab:robust-covariate-adj-table}Robustness Check 1: Sensitivity to Covariate Adjustment vs. Region Fixed Effects}
\centering
\resizebox{\ifdim\width>\linewidth\linewidth\else\width\fi}{!}{
\fontsize{9}{11}\selectfont
\begin{tabular}[t]{>{\raggedright\arraybackslash}p{6em}cccccccc}
\toprule
\multicolumn{1}{c}{ } & \multicolumn{4}{c}{Wave 2 Models} & \multicolumn{4}{c}{Wave 3 Models} \\
\cmidrule(l{3pt}r{3pt}){2-5} \cmidrule(l{3pt}r{3pt}){6-9}
  & (2a) W2 DiD Wgt (FE) & (R1a) W2 DiD Wgt (Cov) & (2b) W2 DiD Unw (FE) & (R1b) W2 DiD Unw (Cov) & (4a) W3 DiD Wgt (FE) & (R1c) W3 DiD Wgt (Cov) & (4b) W3 DiD Unw (FE) & (R1d) W3 DiD Unw (Cov)\\
\midrule
\cellcolor{gray!10}{Treat assignment} & \cellcolor{gray!10}{-0.014} & \cellcolor{gray!10}{-0.030} & \cellcolor{gray!10}{-0.001} & \cellcolor{gray!10}{-0.018} & \cellcolor{gray!10}{-0.061} & \cellcolor{gray!10}{-0.062} & \cellcolor{gray!10}{-0.037} & \cellcolor{gray!10}{-0.039}\\
 & (0.082) & (0.082) & (0.081) & (0.081) & (0.094) & (0.094) & (0.093) & (0.093)\\
\cellcolor{gray!10}{Post-treatment} & \cellcolor{gray!10}{-0.011} & \cellcolor{gray!10}{-0.012} & \cellcolor{gray!10}{-0.013} & \cellcolor{gray!10}{-0.013} & \cellcolor{gray!10}{-0.060} & \cellcolor{gray!10}{-0.059} & \cellcolor{gray!10}{-0.074} & \cellcolor{gray!10}{-0.073}\\
 & (0.080) & (0.080) & (0.079) & (0.079) & (0.093) & (0.093) & (0.092) & (0.092)\\
\cellcolor{gray!10}{Treat x Post} & \cellcolor{gray!10}{-0.301**} & \cellcolor{gray!10}{-0.299**} & \cellcolor{gray!10}{-0.298**} & \cellcolor{gray!10}{-0.296**} & \cellcolor{gray!10}{-0.128} & \cellcolor{gray!10}{-0.128} & \cellcolor{gray!10}{-0.130} & \cellcolor{gray!10}{-0.130}\\
 & (0.116) & (0.115) & (0.115) & (0.114) & (0.132) & (0.132) & (0.131) & (0.131)\\
\cellcolor{gray!10}{Age} & \cellcolor{gray!10}{} & \cellcolor{gray!10}{-0.014*} & \cellcolor{gray!10}{} & \cellcolor{gray!10}{-0.015*} & \cellcolor{gray!10}{} & \cellcolor{gray!10}{-0.018*} & \cellcolor{gray!10}{} & \cellcolor{gray!10}{-0.019*}\\
 &  & (0.007) &  & (0.007) &  & (0.008) &  & (0.008)\\
\cellcolor{gray!10}{Education} & \cellcolor{gray!10}{} & \cellcolor{gray!10}{0.125†} & \cellcolor{gray!10}{} & \cellcolor{gray!10}{0.129*} & \cellcolor{gray!10}{} & \cellcolor{gray!10}{-0.050} & \cellcolor{gray!10}{} & \cellcolor{gray!10}{-0.050}\\
 &  & (0.066) &  & (0.066) &  & (0.079) &  & (0.077)\\
\cellcolor{gray!10}{Male} & \cellcolor{gray!10}{} & \cellcolor{gray!10}{-0.159**} & \cellcolor{gray!10}{} & \cellcolor{gray!10}{-0.158**} & \cellcolor{gray!10}{} & \cellcolor{gray!10}{-0.078} & \cellcolor{gray!10}{} & \cellcolor{gray!10}{-0.079}\\
 &  & (0.060) &  & (0.059) &  & (0.069) &  & (0.068)\\
\cellcolor{gray!10}{Score} & \cellcolor{gray!10}{} & \cellcolor{gray!10}{-0.004†} & \cellcolor{gray!10}{} & \cellcolor{gray!10}{-0.005*} & \cellcolor{gray!10}{} & \cellcolor{gray!10}{-0.004} & \cellcolor{gray!10}{} & \cellcolor{gray!10}{-0.004}\\
 &  & (0.003) &  & (0.003) &  & (0.003) &  & (0.003)\\
\midrule
\cellcolor{gray!10}{Observations} & \cellcolor{gray!10}{1365} & \cellcolor{gray!10}{1365} & \cellcolor{gray!10}{1398} & \cellcolor{gray!10}{1398} & \cellcolor{gray!10}{1053} & \cellcolor{gray!10}{1053} & \cellcolor{gray!10}{1081} & \cellcolor{gray!10}{1081}\\
R$^{2}$ & 0.023 & 0.030 & 0.022 & 0.031 & 0.012 & 0.018 & 0.012 & 0.017\\
\cellcolor{gray!10}{Adj. R$^{2}$} & \cellcolor{gray!10}{0.019} & \cellcolor{gray!10}{0.025} & \cellcolor{gray!10}{0.018} & \cellcolor{gray!10}{0.026} & \cellcolor{gray!10}{0.007} & \cellcolor{gray!10}{0.011} & \cellcolor{gray!10}{0.006} & \cellcolor{gray!10}{0.011}\\
\bottomrule
\multicolumn{9}{l}{\rule{0pt}{1em}† p $<$ 0.1, * p $<$ 0.05, ** p $<$ 0.01, *** p $<$ 0.001}\\
\multicolumn{9}{l}{\rule{0pt}{1em}Standard errors in parentheses.}\\
\multicolumn{9}{l}{\rule{0pt}{1em}FE models include Region Fixed Effects.}\\
\multicolumn{9}{l}{\rule{0pt}{1em}Cov models replace Region FE with baseline Age, Education, Male, Score.}\\
\end{tabular}}
\end{table}

\textbf{Summary of Covariate Adjustment Check:}
Difference-in-Differences (DiD) estimation relies on controlling for
confounding factors to isolate the causal effect of the treatment. The
primary analysis used regional fixed effects to account for unobserved
time-invariant differences between regions. This robustness check
assesses sensitivity to an alternative control strategy: replacing
region fixed effects with observable baseline individual characteristics
(\texttt{age}, \texttt{educ}, \texttt{male}, \texttt{Score}) that might
otherwise lead to omitted variable bias. The results show that the
estimated DiD treatment effect (\texttt{Treat\ x\ Post}) remains
virtually unchanged in both magnitude and statistical significance
across waves, regardless of whether regional fixed effects or these
individual covariates are used. This strengthens confidence that the
main findings are robust to different approaches for controlling
potential confounders.

\subsubsection{2. Ordered Logit/Probit
Models}\label{ordered-logitprobit-models}

The outcome variable, migration aspiration (\texttt{mg.asp}), is
measured on a 5-point ordinal scale. OLS assumes a continuous,
interval-level variable. Here, we use Ordered Logit models
(specifically, proportional odds logistic regression via
\texttt{MASS::polr}) which explicitly model the ordinal nature of the
outcome. We estimate weighted DiD models similar to the main analysis.

\begin{verbatim}
## Generating table for Robustness Check 2...
\end{verbatim}

\begin{table}[!h]
\centering\centering
\caption{\label{tab:robust-ologit-table}Robustness Check 2: Ordered Logit vs. OLS for Migration Aspiration}
\centering
\resizebox{\ifdim\width>\linewidth\linewidth\else\width\fi}{!}{
\fontsize{10}{12}\selectfont
\begin{tabular}[t]{lcccc}
\toprule
\multicolumn{1}{c}{ } & \multicolumn{2}{c}{Wave 2 Comparison} & \multicolumn{2}{c}{Wave 3 Comparison} \\
\cmidrule(l{3pt}r{3pt}){2-3} \cmidrule(l{3pt}r{3pt}){4-5}
  & (R2a) OrdLogit W2 DiD Wgt & (R2b) OrdLogit W3 DiD Wgt & (2a) OLS W2 DiD Wgt & (4a) OLS W3 DiD Wgt\\
\midrule
\cellcolor{gray!10}{Treat assignment} & \cellcolor{gray!10}{-0.026} & \cellcolor{gray!10}{-0.106} & \cellcolor{gray!10}{-0.014} & \cellcolor{gray!10}{-0.061}\\
 & (0.132) & (0.132) & (0.082) & (0.094)\\
\cellcolor{gray!10}{Post-treatment} & \cellcolor{gray!10}{0.035} & \cellcolor{gray!10}{-0.096} & \cellcolor{gray!10}{-0.011} & \cellcolor{gray!10}{-0.060}\\
 & (0.131) & (0.131) & (0.080) & (0.093)\\
\cellcolor{gray!10}{Treat x Post} & \cellcolor{gray!10}{-0.548**} & \cellcolor{gray!10}{-0.217} & \cellcolor{gray!10}{-0.301**} & \cellcolor{gray!10}{-0.128}\\
 & (0.189) & (0.187) & (0.116) & (0.132)\\
\midrule
\cellcolor{gray!10}{Observations} & \cellcolor{gray!10}{1499} & \cellcolor{gray!10}{1509} & \cellcolor{gray!10}{1365} & \cellcolor{gray!10}{1053}\\
\bottomrule
\multicolumn{5}{l}{\rule{0pt}{1em}† p $<$ 0.1, * p $<$ 0.05, ** p $<$ 0.01, *** p $<$ 0.001}\\
\multicolumn{5}{l}{\rule{0pt}{1em}Standard errors in parentheses.}\\
\multicolumn{5}{l}{\rule{0pt}{1em}OLS models are the original weighted DiD specifications.}\\
\multicolumn{5}{l}{\rule{0pt}{1em}OrdLogit models use weighted proportional odds logistic regression.}\\
\multicolumn{5}{l}{\rule{0pt}{1em}All models include regional fixed effects.}\\
\end{tabular}}
\end{table}

\textbf{Summary of Ordered Logit Check:} While linear models like OLS
are commonly used for estimating causal effects in DiD settings, the
migration aspiration outcome is measured on an ordinal scale. Such
models assume a linear relationship which might not hold. This check
employs an Ordered Logit model, which respects the ordinal nature of the
dependent variable without imposing the same linearity assumption as
OLS. Reassuringly, the pattern of results aligns with the OLS findings:
the key DiD interaction term (\texttt{Treat\ x\ Post}) is negative and
statistically significant in Wave 2, but insignificant in Wave 3. This
indicates that the substantive conclusion about the program's transient
negative effect on migration aspirations is robust and not merely an
artifact of the linear functional form assumed by OLS.

\subsubsection{3. Placebo Test (Pre-Treatment
Trends)}\label{placebo-test-pre-treatment-trends}

A placebo test typically involves examining trends \emph{before} the
intervention occurs. We would need at least two data points
\emph{before} the treatment began (e.g., Wave -1 and Wave 0) to run a
DiD analysis where no effect is expected.

Based on the study timeline (\textbf{Figure 1} in the paper), data
collection started with Wave 1 in Dec 2020/Jan 2021, \emph{just before}
the program began in Feb 2021. There are no pre-treatment waves
available in the provided data (\texttt{mig\_long} only contains waves
1, 2, 3).

\textbf{Therefore, a pre-treatment trend placebo test using the DiD
framework is not feasible with the available data.} The baseline balance
check (Table A1) serves as the primary method to assess pre-treatment
comparability in this RCT context.

\subsubsection{4. Heterogeneity of Treatment Effects
(HTE)}\label{heterogeneity-of-treatment-effects-hte}

Does the program effect differ for specific subgroups? We test this by
interacting the treatment effect (\texttt{treat.dum:post.dum} in the DiD
model) with key baseline characteristics available in
\texttt{mig\_long}: \texttt{male}, \texttt{educ}, and
\texttt{bs.exp.now} (currently has business). We focus on the weighted
Wave 2 DiD model where the main effect was strongest.

\begin{verbatim}
## Generating table for Robustness Check 4...
\end{verbatim}

\begin{table}[!h]
\centering\centering
\caption{\label{tab:robust-hte-table}Robustness Check 4: Heterogeneity of Treatment Effects (Wave 2)}
\centering
\fontsize{9}{11}\selectfont
\begin{tabular}[t]{>{\raggedright\arraybackslash}p{12em}cccc}
\toprule
  & Interact: Male & Interact: Education & Interact: Has Business & (2a) Original W2 DiD Wgt\\
\midrule
\cellcolor{gray!10}{Treat x Post (Main Effect)} & \cellcolor{gray!10}{-0.177} & \cellcolor{gray!10}{-0.226} & \cellcolor{gray!10}{-0.453**} & \cellcolor{gray!10}{-0.301**}\\
 & (0.185) & (0.587) & (0.167) & (0.116)\\
\cellcolor{gray!10}{Treat x Post x Male} & \cellcolor{gray!10}{-0.204} & \cellcolor{gray!10}{} & \cellcolor{gray!10}{} & \cellcolor{gray!10}{}\\
 & (0.237) &  &  & \\
\cellcolor{gray!10}{Treat x Post x Education} & \cellcolor{gray!10}{} & \cellcolor{gray!10}{-0.035} & \cellcolor{gray!10}{} & \cellcolor{gray!10}{}\\
 &  & (0.262) &  & \\
\cellcolor{gray!10}{Treat x Post x Has Business} & \cellcolor{gray!10}{} & \cellcolor{gray!10}{} & \cellcolor{gray!10}{0.293} & \cellcolor{gray!10}{}\\
 &  &  & (0.231) & \\
\midrule
\cellcolor{gray!10}{Observations} & \cellcolor{gray!10}{1365} & \cellcolor{gray!10}{1365} & \cellcolor{gray!10}{1365} & \cellcolor{gray!10}{1365}\\
R$^{2}$ & 0.031 & 0.026 & 0.036 & 0.023\\
\bottomrule
\multicolumn{5}{l}{\rule{0pt}{1em}† p $<$ 0.1, * p $<$ 0.05, ** p $<$ 0.01, *** p $<$ 0.001}\\
\multicolumn{5}{l}{\rule{0pt}{1em}Standard errors in parentheses.}\\
\multicolumn{5}{l}{\rule{0pt}{1em}Models are weighted Wave 2 DiD specifications with Region FE.}\\
\multicolumn{5}{l}{\rule{0pt}{1em}Each interaction model adds a triple interaction term using baseline moderators.}\\
\multicolumn{5}{l}{\rule{0pt}{1em}Significance on the triple interaction suggests the treatment effect differs by the moderator.}\\
\end{tabular}
\end{table}

\textbf{Summary of HTE Check:} Causal effects often exhibit
heterogeneity, meaning the impact of an intervention may differ across
subgroups. The main DiD model estimates an average effect. This check
explores potential heterogeneity by testing if the estimated treatment
effect interacts significantly with observable baseline characteristics
(\texttt{male}, \texttt{educ}, \texttt{bs.exp.now}). Specifically, we
examine the triple interaction term
(\texttt{Treat\ x\ Post\ x\ Moderator}) in the weighted Wave 2 DiD
model. None of these interaction terms were statistically significant,
suggesting that the average short-term treatment effect identified in
the main analysis does not significantly differ across these specific
subgroups. The estimated average dampening effect on migration
aspirations appears relatively homogeneous in the short term with
respect to gender, education, and prior business experience.

\subsubsection{5. Alternative HTE Methods (Future
Work)}\label{alternative-hte-methods-future-work}

More advanced methods like Causal Forests (implemented in the
\texttt{grf} package in R) can explore treatment effect heterogeneity
more systematically without pre-specifying interaction terms. These
methods use machine learning techniques to identify subgroups based on
multiple baseline covariates that experience differential treatment
effects. Implementing causal forests is beyond the scope of this initial
replication but represents a potential avenue for future investigation
to gain deeper insights into \emph{who} benefits most (or least) from
this type of intervention.

\subsubsection{5. Clustered Standard
Errors}\label{clustered-standard-errors}

Given that randomization was blocked by regional training site, we
revisit our key DiD models with cluster-robust standard errors to
account for potential correlation of errors within regions. We first
assess the degree of intraclass correlation (ICC) within regions to
determine whether clustering is justified.

\begin{table}[H]
\centering
\caption{\label{tab:robust-cluster-icc}Intraclass Correlation Coefficients (ICC) by Region}
\centering
\fontsize{9}{11}\selectfont
\begin{tabular}[t]{lr}
\toprule
Analysis & ICC\\
\midrule
\cellcolor{gray!10}{Wave 2} & \cellcolor{gray!10}{0.009}\\
Wave 3 & 0.004\\
\bottomrule
\multicolumn{2}{l}{\rule{0pt}{1em}\textit{Note: } ICC measures the proportion of variance attributable to regional clusters.}\\
\end{tabular}
\end{table}

Based on the ICC values above, clustering the standard errors appears to
be \textbf{not strongly justified}. Values above 0.05 typically warrant
cluster adjustment, while values below 0.01 suggest minimal clustering
effects. Indeed, our ICC values (0.009 for Wave 2, 0.004 for Wave 3)
indicate very little within-region correlation in migration aspirations.

However, a curious disconnect emerges when we apply cluster-robust
standard errors: despite minimal ICC values, we observe substantial
increases in standard errors. This apparent contradiction stems from a
well-known methodological challenge in econometrics: the ``few clusters
problem.'' With only three regions in our data (Greater Banjul Area,
Lower River Region, Upper River Region), conventional cluster-robust
standard errors become unreliable and typically biased upward,
regardless of actual within-cluster correlation. Econometric literature
recommends at least 10-15 clusters (preferably 30+) for reliable
cluster-robust inference.

Nevertheless, we proceed with implementing cluster-robust standard
errors to illustrate this methodological issue, while cautioning that
the resulting significance changes may be artifacts of applying cluster
adjustment with too few clusters rather than evidence of meaningful
clustering effects.

\begin{table}[H]
\centering
\caption{\label{tab:robust-clustered-se-simple}Effect of Regional Clustering on Standard Errors (DiD Treatment Effect)}
\centering
\fontsize{9}{11}\selectfont
\begin{tabular}[t]{lrrrrrr}
\toprule
Model & Coefficient & Standard SE & Standard p & Clustered SE & Clustered p & SE Ratio\\
\midrule
\cellcolor{gray!10}{Wave 2 DiD} & \cellcolor{gray!10}{-0.301} & \cellcolor{gray!10}{0.116} & \cellcolor{gray!10}{0.009} & \cellcolor{gray!10}{0.157} & \cellcolor{gray!10}{0.056} & \cellcolor{gray!10}{1.36}\\
Wave 3 DiD & -0.128 & 0.132 & 0.335 & 0.239 & 0.594 & 1.81\\
\bottomrule
\multicolumn{7}{l}{\rule{0pt}{1em}\textit{Note: } SE Ratio = Clustered SE / Standard SE. Wave 2 p-value changes from 0.009 to 0.056.}\\
\end{tabular}
\end{table}

\begin{table}[H]
\centering
\caption{\label{tab:robust-clustered-se-simple}Detailed Comparison for Wave 2 DiD 'Treat x Post' Coefficient}
\centering
\fontsize{9}{11}\selectfont
\begin{tabular}[t]{lrrrr}
\toprule
Estimator & Estimate & Std.Error & t.value & p.value\\
\midrule
\cellcolor{gray!10}{Conventional SE} & \cellcolor{gray!10}{-0.301} & \cellcolor{gray!10}{0.116} & \cellcolor{gray!10}{-2.60} & \cellcolor{gray!10}{0.009}\\
Cluster-robust SE & -0.301 & 0.157 & -1.91 & 0.056\\
\bottomrule
\end{tabular}
\end{table}

\textbf{Summary of Clustered Standard Errors Check:} This robustness
check addresses an important methodological concern regarding the
clustered nature of the randomization design. As shown in the table,
when we account for potential correlation of errors within regions by
using cluster-robust standard errors, the statistical significance of
our key findings changes.

For the Wave 2 model, the key interaction term ``Treat × Post''
coefficient remains unchanged at approximately -0.301, indicating the
program reduced migration aspirations by 0.3 points on the 5-point scale
in the short term. However, the clustered standard error is larger than
the conventional standard error, increasing the p-value from around
0.017 to 0.056. This means the short-term effect becomes only marginally
significant at the p\textless0.1 level, rather than significant at the
conventional p\textless0.05 level as in our main analysis.

This finding is particularly important given the original paper's
emphasis on this short-term effect as primary evidence that foreign aid
can reduce migration aspirations. The more conservative standard errors
provided by clustering indicate that we should be more cautious about
this conclusion, as the evidence for the short-term effect is weaker
than previously suggested.

The Wave 3 results, which were already insignificant in our main
analysis, remain insignificant with clustered standard errors, further
supporting the conclusion about the transient nature of the program's
effects.

\begin{verbatim}
## Robustness check code sections added/corrected.
\end{verbatim}

\subsection{Discussion and Comparison with Original
Paper}\label{discussion-and-comparison-with-original-paper}

Our replication study largely confirms the core findings from Simon,
Schwartz, and Hudson's original paper. Here we summarize the key
comparisons between our replication results and the original study:

\subsubsection{Confirmation of Key
Findings}\label{confirmation-of-key-findings}

\begin{enumerate}
\def\labelenumi{\arabic{enumi}.}
\item
  Business outcomes: Our Table 1 replicates the pattern of significant
  positive treatment effects on all business outcomes except for
  business closure. Treatment group participants were substantially more
  likely to open businesses, make purchases, hire employees, and earn
  profits compared to the control group.
\item
  **Short-term migration aspirations: Our weighted DiD models in Table
  2a yield almost identical treatment coefficients to the original paper
  (approximately -0.301), confirming the program's significant negative
  effect on migration aspirations in the short term (Wave 2).
\item
  Fading long-term effects**: Like the original paper, our results in
  Table 2b show the treatment effects on migration aspirations diminish
  by Wave 3 and lose statistical significance, supporting the authors'
  conclusion about the temporary nature of the intervention's impact.
\item
  Mediation through instrumental place attachment: Our Figures 2 and 3
  corroborate that the program worked primarily through instrumental
  place attachment (``Financial success at home'') rather than through
  self-efficacy or self-sufficiency, aligning with the authors'
  theoretical framework.
\end{enumerate}

\subsubsection{Methodological
Enhancements}\label{methodological-enhancements}

Our replication extends the original analysis with several
methodological refinements:

\begin{enumerate}
\def\labelenumi{\arabic{enumi}.}
\item
  Weighting procedure robustness: By comparing weighted and unweighted
  DiD models (Tables 2a, 2b), we demonstrate that the main results are
  virtually identical regardless of whether inverse probability
  weighting is applied. This suggests that while methodologically sound,
  the original authors' complex weighting scheme had minimal practical
  impact on the estimates.
\item
  Alternative control strategies: Our covariate adjustment robustness
  check shows that replacing region fixed effects with individual-level
  covariates does not meaningfully alter the treatment effect estimates,
  strengthening confidence in the robustness of the results.
\item
  Ordered logit models: By modeling the ordinal nature of the migration
  aspiration outcome directly, we confirm the original findings aren't
  artifacts of treating an ordinal scale as continuous in the OLS
  models.
\end{enumerate}

\subsubsection{Treatment Effect
Heterogeneity}\label{treatment-effect-heterogeneity}

Our heterogeneity analysis finds mostly insignificant interaction terms,
suggesting the program effects were relatively homogeneous across
gender, education levels, and prior business experience. This finding
augments the original paper by demonstrating broader applicability of
the intervention across participant characteristics.

\subsubsection{Theoretical Implications}\label{theoretical-implications}

Our results reinforce the authors' theoretical focus on instrumental
place attachment as the primary mechanism through which foreign aid can
reduce migration aspirations. The finding that neither self-efficacy nor
self-sufficiency significantly mediated migration aspirations, despite
being improved by the program, supports the authors' contention that
these traditional development outcomes are insufficient to change
migration attitudes without also increasing place attachment.

\subsubsection{Limitations and Practical
Considerations}\label{limitations-and-practical-considerations}

The transient nature of the program's effects on migration aspirations,
with significant impacts in Wave 2 that disappear by Wave 3, aligns with
the qualitative insights from the original paper's interviews suggesting
that maintaining instrumental place attachment requires ongoing support
networks and continued assistance navigating bureaucratic barriers.

In conclusion, our replication supports the original paper's findings
while providing additional evidence on robustness and generalizability.
The results suggest that foreign aid interventions can temporarily
reduce migration aspirations through instrumental place attachment, but
sustaining these effects likely requires more persistent, structural
interventions that address the broader institutional environment.

\subsection{Comparative Analysis of Treatment
Effects}\label{comparative-analysis-of-treatment-effects}

Our replication closely mirrors the treatment effects reported in the
original paper, confirming their robustness. In the original study,
Simon et al.~found that the program reduced migration aspirations by
approximately 0.30 points on their 5-point scale in the short term (Wave
2), with high statistical significance (p \textless{} 0.01). Our
replicated Wave 2 DiD coefficient of -0.301 (p \textless{} 0.01) matches
this finding almost exactly. Similarly, our Wave 3 DiD coefficient of
-0.128 (not statistically significant) aligns with their reported
finding that the effect ``faded'' over time.

This consistency is particularly noteworthy given three methodological
differences in our approach:

\begin{enumerate}
\def\labelenumi{\arabic{enumi}.}
\tightlist
\item
  We separated Wave 2 and Wave 3 results into different tables for
  clarity, while the original paper presented them together
\item
  We employed alternative statistical approaches including unweighted
  models and alternative control strategies
\item
  We applied multiple testing corrections (Hochberg method) to adjust
  for familywise error rate
\end{enumerate}

The fact that our coefficients remain virtually identical across these
different specifications underscores the original finding's robustness.
The diminishing effect over time (from -0.301 to -0.128) also validates
the authors' qualitative findings from interviews, which suggested that
instrumental place attachment requires ongoing support networks and
continued assistance navigating bureaucratic barriers to be sustained.

\end{document}
